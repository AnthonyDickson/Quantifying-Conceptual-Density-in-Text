\documentclass[12pt]{article}
\usepackage[english]{babel}
\usepackage[utf8]{inputenc}
\usepackage{fancyhdr}
\usepackage{enumitem}
 
\pagestyle{fancy}
\fancyhf{}

\lhead{Anthony Dickson (3348967)} 
\rhead{COSC480 Aims and Objectives} 

\rfoot{\today}


\begin{document}

%Your project title and supervisor(s)
\noindent{\textsc{Quantifying conceptual density in text}} \\
\noindent{
        Supervisors:
        Anthony Robins,
        Alistair Knott
}

\paragraph{Aims}
Conceptual Density describes the degree to which concepts are interdependent, or to what degree they interact with eachother. This idea is related to (intrinsic) cognitive load, where understanding and aquisition of new skills \& knowledge may be hindered by an increasing degree of interdependence between concepts. There is a sizeable amount of literature on cognitive load theory, however it remains difficult to objectively measure this phenomenon. We aim to establish a method of quantifying the conceptual density in text corpora.

\paragraph{Objectives}
\begin{itemize}[noitemsep]
\item Background reading on: 

	\begin{itemize}[noitemsep]
		\item Conceptual Density
		\item Hypertext markers
		\item Textual Signposting
		\item Topic Segmentation
		\item Other relevant topics
	\end{itemize}

\item Identify suitable text corpora
\item Design experiments to test published natural language processing methods on the above corpora
\item Develop, implement, and test systems for quantifying conceputal density.
\end{itemize}

\paragraph{Timeline} 
\begin{itemize}[noitemsep]
\item Mar-Apr: Background reading
\item Apr-May: Identifying text corpora
\item Jun: Designing experiments \& preliminary results (Interim report)
\item Jul-Aug: Testing published methods on corpora
\item Sep-Oct: Developing, implementing, and testing systems for quantifying conceptual density (Final report).
\end{itemize}

\end{document}
