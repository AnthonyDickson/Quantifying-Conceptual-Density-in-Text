\documentclass[12pt]{article}
\usepackage[english]{babel}
\usepackage[utf8]{inputenc}
\usepackage{fancyhdr}
\usepackage{enumitem}
 
\pagestyle{fancy}
\fancyhf{}
%Change COSC4x0 to COSC480 or COSC490
\rhead{COSC4x0 Aims and Objectives} 

%Your name and student ID
\lhead{John Smith (555555)} 

\rfoot{\today}


\begin{document}

%Your project title and supervisor(s)
\noindent{\textsc{Project title}} \\
\noindent{
        Supervisor(s):
        James Smith,
        Frank Lee
}

\paragraph{Aims}
Here you are describing the long-term goal of the project.  What do you want to achieve by the end?  What is the ultimate goal of this work?  For example, the primary aim of this document is to have students produce suitable aims and objectives for their COSC480/490 projects.  While the aims and objectives documents is not an assessed deliverable, a clear definition of what is to be done, and a bit of planning of how it is to be accomplished, is paramount to the project's success.  

\paragraph{Objectives}
Objectives list the milestones that you need to achieve in order to achieve the projects aim(s).  It's a rough plan for what needs to happen in what order.  It's best to list the objective in bullet point form.  For many projects the objectives might follow a pattern like this:    
\begin{itemize}[noitemsep]
\item background reading; going through the literature; learning about the research field;
\item setting up of some kind of system for the project; getting the environment for experiments working;
\item conducting preliminary experiments; implementation of a basic/simple approach; producing base case results;
\item trying method 1; recording the results;
\item trying method 2; recording the results.
\end{itemize}

\paragraph{Timeline} It's a good idea to set up a bit of a timeline for the project -- without it it's very hard to gauge your progress.  What you put here is not written in stone.  In fact, it's not unusual for the aims and objectives to get revised at the half way point through the project.
\begin{itemize}[noitemsep]
\item Mar: doing the background;
\item Mar-Apr: system setup;
\item May-Jun: preliminary results (Interim report);
\item Jul-Aug: working on method 1;
\item Sep-Oct: workin on method 2 (Final report).
\end{itemize}

\noindent
The aims and objectives should fit in 1-2 pages.
\end{document}
