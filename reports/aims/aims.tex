\documentclass[12pt]{article}
\usepackage[english]{babel}
\usepackage[utf8]{inputenc}
\usepackage{fancyhdr}
\usepackage{enumitem}
 
\pagestyle{fancy}
\fancyhf{}

\lhead{Anthony Dickson (3348967)} 
\rhead{COSC480 Aims and Objectives} 

\rfoot{\today}


\begin{document}

%Your project title and supervisor(s)
\noindent{\textsc{Quantifying conceptual density in text}} \\
\noindent{
        Supervisors:
        Anthony Robins,
        Alistair Knott
}

\paragraph{Aims}
Conceptual Density describes the degree to which concepts in an extended text (such as the textbook for a university course) are interdependent, or to what degree they interact with each other. There is a hypothesis that text documents with high conceptual density are harder for students to process, because of the high interdependence between concepts. We aim to establish a method of quantifying this idea of conceptual density in text corpora.

\paragraph{Objectives}
\begin{itemize}[noitemsep]
\item Background reading on: 

	\begin{itemize}[noitemsep]
		\item Conceptual Density
		\item Cognitive Load
		\item Hypertext Markers
		\item Textual Signposting
		\item Topic Segmentation
		\item Other relevant topics
	\end{itemize}

\item Identify suitable text corpora
\item Design and implement experiments to test published natural language processing methods on the above corpora
\item If time allows, develop, implement, and test new methods for quantifying conceptual density.
\end{itemize}

\paragraph{Timeline} 
\begin{itemize}[noitemsep]
\item Mar-Apr: Background reading
\item Apr-May: Identifying text corpora
\item Jun: Designing experiments \& preliminary results (Interim report)
\item Jul-Aug: Implementing experiments
\item Sep-Oct: Developing, implementing, and testing new methods for quantifying conceptual density (Final report).
\end{itemize}

\end{document}
