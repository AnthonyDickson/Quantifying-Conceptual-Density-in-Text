%%
%% Author: anthony
%% 15/05/19
%%

% Preamble
\documentclass[11pt]{article}

% Packages
\usepackage{amsmath}


\newcommand{\entity}[1]{{#1}}

% Document
\begin{document}

    \title{Top 5 Secrets for Bug-Free Code. Number 5 will shock you!}
    \maketitle

    \section{Introduction}
    Nobody likes \entity{bugs} in their \entity{code}. In this awesome document
     I will introduce the top 5 secrets to producing bug-free code.

    \section{It is a Feature}
    It is not a \entity{bug}, it is a \entity{feature}! My code does exactly
    what I intended it to do! This motto will keep you motivated and encouraged
    to keep on being that programming rock star that everyone aspires to become.

    \section{Unit Testing}
    Sick of failing \entity{unit tests}? Just change all of your tests to \verb|assert(True)|
    and voil\`a! Now you have super fast \entity{unit tests} that all pass!

    \section{Use Comments}
    We all know that it is common knowledge that fixing one \entity{bug} will introduce
    many more bugs. So why fix bugs if they are just going to introduce more?
    So let's just \entity{comment} out the code that causes the bug (This may or may
    not have a cascading effect).

    \section{A loop hole}
    A quirky \entity{loop hole} is to simply write no \entity{code}. Code that does not
    exist cannot have bugs!

    \section{Write Good Code}
    The best way for writing \entity{bug-free code} is to simply write
    \entity{bug-free code} from the get go. Why bother with debugging and
    fixing buggy code when you can just write perfect code? Easy!


\end{document}
